% Options for packages loaded elsewhere
\PassOptionsToPackage{unicode}{hyperref}
\PassOptionsToPackage{hyphens}{url}
%
\documentclass[
]{article}
\usepackage{amsmath,amssymb}
\usepackage{lmodern}
\usepackage{ifxetex,ifluatex}
\ifnum 0\ifxetex 1\fi\ifluatex 1\fi=0 % if pdftex
  \usepackage[T1]{fontenc}
  \usepackage[utf8]{inputenc}
  \usepackage{textcomp} % provide euro and other symbols
\else % if luatex or xetex
  \usepackage{unicode-math}
  \defaultfontfeatures{Scale=MatchLowercase}
  \defaultfontfeatures[\rmfamily]{Ligatures=TeX,Scale=1}
\fi
% Use upquote if available, for straight quotes in verbatim environments
\IfFileExists{upquote.sty}{\usepackage{upquote}}{}
\IfFileExists{microtype.sty}{% use microtype if available
  \usepackage[]{microtype}
  \UseMicrotypeSet[protrusion]{basicmath} % disable protrusion for tt fonts
}{}
\makeatletter
\@ifundefined{KOMAClassName}{% if non-KOMA class
  \IfFileExists{parskip.sty}{%
    \usepackage{parskip}
  }{% else
    \setlength{\parindent}{0pt}
    \setlength{\parskip}{6pt plus 2pt minus 1pt}}
}{% if KOMA class
  \KOMAoptions{parskip=half}}
\makeatother
\usepackage{xcolor}
\IfFileExists{xurl.sty}{\usepackage{xurl}}{} % add URL line breaks if available
\IfFileExists{bookmark.sty}{\usepackage{bookmark}}{\usepackage{hyperref}}
\hypersetup{
  pdftitle={Latihan 5},
  pdfauthor={Hamzah Abdulloh},
  hidelinks,
  pdfcreator={LaTeX via pandoc}}
\urlstyle{same} % disable monospaced font for URLs
\usepackage[margin=1in]{geometry}
\usepackage{color}
\usepackage{fancyvrb}
\newcommand{\VerbBar}{|}
\newcommand{\VERB}{\Verb[commandchars=\\\{\}]}
\DefineVerbatimEnvironment{Highlighting}{Verbatim}{commandchars=\\\{\}}
% Add ',fontsize=\small' for more characters per line
\usepackage{framed}
\definecolor{shadecolor}{RGB}{248,248,248}
\newenvironment{Shaded}{\begin{snugshade}}{\end{snugshade}}
\newcommand{\AlertTok}[1]{\textcolor[rgb]{0.94,0.16,0.16}{#1}}
\newcommand{\AnnotationTok}[1]{\textcolor[rgb]{0.56,0.35,0.01}{\textbf{\textit{#1}}}}
\newcommand{\AttributeTok}[1]{\textcolor[rgb]{0.77,0.63,0.00}{#1}}
\newcommand{\BaseNTok}[1]{\textcolor[rgb]{0.00,0.00,0.81}{#1}}
\newcommand{\BuiltInTok}[1]{#1}
\newcommand{\CharTok}[1]{\textcolor[rgb]{0.31,0.60,0.02}{#1}}
\newcommand{\CommentTok}[1]{\textcolor[rgb]{0.56,0.35,0.01}{\textit{#1}}}
\newcommand{\CommentVarTok}[1]{\textcolor[rgb]{0.56,0.35,0.01}{\textbf{\textit{#1}}}}
\newcommand{\ConstantTok}[1]{\textcolor[rgb]{0.00,0.00,0.00}{#1}}
\newcommand{\ControlFlowTok}[1]{\textcolor[rgb]{0.13,0.29,0.53}{\textbf{#1}}}
\newcommand{\DataTypeTok}[1]{\textcolor[rgb]{0.13,0.29,0.53}{#1}}
\newcommand{\DecValTok}[1]{\textcolor[rgb]{0.00,0.00,0.81}{#1}}
\newcommand{\DocumentationTok}[1]{\textcolor[rgb]{0.56,0.35,0.01}{\textbf{\textit{#1}}}}
\newcommand{\ErrorTok}[1]{\textcolor[rgb]{0.64,0.00,0.00}{\textbf{#1}}}
\newcommand{\ExtensionTok}[1]{#1}
\newcommand{\FloatTok}[1]{\textcolor[rgb]{0.00,0.00,0.81}{#1}}
\newcommand{\FunctionTok}[1]{\textcolor[rgb]{0.00,0.00,0.00}{#1}}
\newcommand{\ImportTok}[1]{#1}
\newcommand{\InformationTok}[1]{\textcolor[rgb]{0.56,0.35,0.01}{\textbf{\textit{#1}}}}
\newcommand{\KeywordTok}[1]{\textcolor[rgb]{0.13,0.29,0.53}{\textbf{#1}}}
\newcommand{\NormalTok}[1]{#1}
\newcommand{\OperatorTok}[1]{\textcolor[rgb]{0.81,0.36,0.00}{\textbf{#1}}}
\newcommand{\OtherTok}[1]{\textcolor[rgb]{0.56,0.35,0.01}{#1}}
\newcommand{\PreprocessorTok}[1]{\textcolor[rgb]{0.56,0.35,0.01}{\textit{#1}}}
\newcommand{\RegionMarkerTok}[1]{#1}
\newcommand{\SpecialCharTok}[1]{\textcolor[rgb]{0.00,0.00,0.00}{#1}}
\newcommand{\SpecialStringTok}[1]{\textcolor[rgb]{0.31,0.60,0.02}{#1}}
\newcommand{\StringTok}[1]{\textcolor[rgb]{0.31,0.60,0.02}{#1}}
\newcommand{\VariableTok}[1]{\textcolor[rgb]{0.00,0.00,0.00}{#1}}
\newcommand{\VerbatimStringTok}[1]{\textcolor[rgb]{0.31,0.60,0.02}{#1}}
\newcommand{\WarningTok}[1]{\textcolor[rgb]{0.56,0.35,0.01}{\textbf{\textit{#1}}}}
\usepackage{graphicx}
\makeatletter
\def\maxwidth{\ifdim\Gin@nat@width>\linewidth\linewidth\else\Gin@nat@width\fi}
\def\maxheight{\ifdim\Gin@nat@height>\textheight\textheight\else\Gin@nat@height\fi}
\makeatother
% Scale images if necessary, so that they will not overflow the page
% margins by default, and it is still possible to overwrite the defaults
% using explicit options in \includegraphics[width, height, ...]{}
\setkeys{Gin}{width=\maxwidth,height=\maxheight,keepaspectratio}
% Set default figure placement to htbp
\makeatletter
\def\fps@figure{htbp}
\makeatother
\setlength{\emergencystretch}{3em} % prevent overfull lines
\providecommand{\tightlist}{%
  \setlength{\itemsep}{0pt}\setlength{\parskip}{0pt}}
\setcounter{secnumdepth}{-\maxdimen} % remove section numbering
\ifluatex
  \usepackage{selnolig}  % disable illegal ligatures
\fi

\title{Latihan 5}
\author{Hamzah Abdulloh}
\date{10/27/2021}

\begin{document}
\maketitle

\begin{Shaded}
\begin{Highlighting}[]
\FunctionTok{library}\NormalTok{(dslabs)}
\FunctionTok{data}\NormalTok{(murders) }
\end{Highlighting}
\end{Shaded}

\hypertarget{modul-5}{%
\section{Modul 5}\label{modul-5}}

\begin{enumerate}
\def\labelenumi{\arabic{enumi}.}
\tightlist
\item
  Fungsi nchar dapat digunakan untuk menghitung jumlah karakter dari
  suatu vektor karakter. Buatlah satu baris kode yang akan menyimpan
  hasil komputasi pada variabel `new\_names' dan berisi singkatan nama
  negara ketika jumlah karakternya lebih dari 8 karakter. .
\end{enumerate}

\begin{Shaded}
\begin{Highlighting}[]
\NormalTok{new\_names }\OtherTok{\textless{}{-}} \FunctionTok{ifelse}\NormalTok{(}\FunctionTok{nchar}\NormalTok{(murders}\SpecialCharTok{$}\NormalTok{state) }\SpecialCharTok{\textgreater{}} \DecValTok{8}\NormalTok{, murders}\SpecialCharTok{$}\NormalTok{abb, murders}\SpecialCharTok{$}\NormalTok{state) }
\NormalTok{new\_names}
\end{Highlighting}
\end{Shaded}

\begin{verbatim}
##  [1] "Alabama"  "Alaska"   "Arizona"  "Arkansas" "CA"       "Colorado"
##  [7] "CT"       "Delaware" "DC"       "Florida"  "Georgia"  "Hawaii"  
## [13] "Idaho"    "Illinois" "Indiana"  "Iowa"     "Kansas"   "Kentucky"
## [19] "LA"       "Maine"    "Maryland" "MA"       "Michigan" "MN"      
## [25] "MS"       "Missouri" "Montana"  "Nebraska" "Nevada"   "NH"      
## [31] "NJ"       "NM"       "New York" "NC"       "ND"       "Ohio"    
## [37] "Oklahoma" "Oregon"   "PA"       "RI"       "SC"       "SD"      
## [43] "TN"       "Texas"    "Utah"     "Vermont"  "Virginia" "WA"      
## [49] "WV"       "WI"       "Wyoming"
\end{verbatim}

\begin{enumerate}
\def\labelenumi{\arabic{enumi}.}
\setcounter{enumi}{1}
\tightlist
\item
  Buat fungsi sum\_n yang dapat digunakan untuk menghitung jumlah
  bilangan bulat dari 1 hingga n.~Gunakan pula fungsi ini untuk
  menentukan jumlah bilangan bulat dari 1 hingga 5.000. .
\end{enumerate}

\begin{Shaded}
\begin{Highlighting}[]
\NormalTok{sum\_n }\OtherTok{\textless{}{-}} \ControlFlowTok{function}\NormalTok{(n)\{}
\NormalTok{ x}\OtherTok{\textless{}{-}} \DecValTok{1}\SpecialCharTok{:}\NormalTok{n}
 \FunctionTok{sum}\NormalTok{(x)}
\NormalTok{\}}
\NormalTok{n}\OtherTok{\textless{}{-}}\DecValTok{5000}
\FunctionTok{sum\_n}\NormalTok{(n)}
\end{Highlighting}
\end{Shaded}

\begin{verbatim}
## [1] 12502500
\end{verbatim}

\begin{enumerate}
\def\labelenumi{\arabic{enumi}.}
\setcounter{enumi}{2}
\tightlist
\item
  Buat fungsi compute\_s\_n yang dapat digunakan untuk menghitung jumlah
  Sn = 1\^{}2+ 2\^{}2+ 3\^{}2+. . . n\^{}2. Tampilkan hasil penjumlahan
  ketika n = 10.
\end{enumerate}

\begin{Shaded}
\begin{Highlighting}[]
\NormalTok{compute\_s\_n }\OtherTok{\textless{}{-}} \ControlFlowTok{function}\NormalTok{(n)\{ }
\NormalTok{x }\OtherTok{\textless{}{-}} \DecValTok{1}\SpecialCharTok{:}\NormalTok{n }
\FunctionTok{sum}\NormalTok{(x}\SpecialCharTok{*}\NormalTok{x) }
\NormalTok{\} }
\NormalTok{n}\OtherTok{\textless{}{-}}\DecValTok{10}
\FunctionTok{compute\_s\_n}\NormalTok{(n) }
\end{Highlighting}
\end{Shaded}

\begin{verbatim}
## [1] 385
\end{verbatim}

\begin{enumerate}
\def\labelenumi{\arabic{enumi}.}
\setcounter{enumi}{3}
\tightlist
\item
  Buat vektor numerik kosong dengan nama: s\_n dengan ukuran:25
  menggunakan s\_n \textless- vector (``numeric'', 25). Simpan di hasil
  komputasi S1, S2,. . . S25 menggunakan FOR-LOOP. .
\end{enumerate}

\begin{Shaded}
\begin{Highlighting}[]
\NormalTok{len }\OtherTok{\textless{}{-}} \DecValTok{25} 
\NormalTok{s\_n }\OtherTok{\textless{}{-}} \FunctionTok{vector}\NormalTok{(}\StringTok{"numeric"}\NormalTok{, }\AttributeTok{length =}\NormalTok{ len)}
\ControlFlowTok{for}\NormalTok{(n }\ControlFlowTok{in} \DecValTok{1}\SpecialCharTok{:}\NormalTok{len)\{ }
\NormalTok{s\_n[n] }\OtherTok{\textless{}{-}} \FunctionTok{compute\_s\_n}\NormalTok{(n) }
\NormalTok{\} }
\NormalTok{s\_n}
\end{Highlighting}
\end{Shaded}

\begin{verbatim}
##  [1]    1    5   14   30   55   91  140  204  285  385  506  650  819 1015 1240
## [16] 1496 1785 2109 2470 2870 3311 3795 4324 4900 5525
\end{verbatim}

\begin{enumerate}
\def\labelenumi{\arabic{enumi}.}
\setcounter{enumi}{4}
\tightlist
\item
  Ulangi langkah pada soal no. 4 dan gunakan fugsi sapply.
\end{enumerate}

\begin{Shaded}
\begin{Highlighting}[]
\NormalTok{n }\OtherTok{\textless{}{-}} \DecValTok{1}\SpecialCharTok{:}\DecValTok{25} 
\NormalTok{s\_n }\OtherTok{\textless{}{-}} \FunctionTok{sapply}\NormalTok{(n, compute\_s\_n) }
\NormalTok{s\_n}
\end{Highlighting}
\end{Shaded}

\begin{verbatim}
##  [1]    1    5   14   30   55   91  140  204  285  385  506  650  819 1015 1240
## [16] 1496 1785 2109 2470 2870 3311 3795 4324 4900 5525
\end{verbatim}

\end{document}
