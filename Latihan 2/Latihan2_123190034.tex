% Options for packages loaded elsewhere
\PassOptionsToPackage{unicode}{hyperref}
\PassOptionsToPackage{hyphens}{url}
%
\documentclass[
]{article}
\usepackage{amsmath,amssymb}
\usepackage{lmodern}
\usepackage{ifxetex,ifluatex}
\ifnum 0\ifxetex 1\fi\ifluatex 1\fi=0 % if pdftex
  \usepackage[T1]{fontenc}
  \usepackage[utf8]{inputenc}
  \usepackage{textcomp} % provide euro and other symbols
\else % if luatex or xetex
  \usepackage{unicode-math}
  \defaultfontfeatures{Scale=MatchLowercase}
  \defaultfontfeatures[\rmfamily]{Ligatures=TeX,Scale=1}
\fi
% Use upquote if available, for straight quotes in verbatim environments
\IfFileExists{upquote.sty}{\usepackage{upquote}}{}
\IfFileExists{microtype.sty}{% use microtype if available
  \usepackage[]{microtype}
  \UseMicrotypeSet[protrusion]{basicmath} % disable protrusion for tt fonts
}{}
\makeatletter
\@ifundefined{KOMAClassName}{% if non-KOMA class
  \IfFileExists{parskip.sty}{%
    \usepackage{parskip}
  }{% else
    \setlength{\parindent}{0pt}
    \setlength{\parskip}{6pt plus 2pt minus 1pt}}
}{% if KOMA class
  \KOMAoptions{parskip=half}}
\makeatother
\usepackage{xcolor}
\IfFileExists{xurl.sty}{\usepackage{xurl}}{} % add URL line breaks if available
\IfFileExists{bookmark.sty}{\usepackage{bookmark}}{\usepackage{hyperref}}
\hypersetup{
  pdftitle={Latihan 2},
  pdfauthor={Hamzah Abdulloh},
  hidelinks,
  pdfcreator={LaTeX via pandoc}}
\urlstyle{same} % disable monospaced font for URLs
\usepackage[margin=1in]{geometry}
\usepackage{color}
\usepackage{fancyvrb}
\newcommand{\VerbBar}{|}
\newcommand{\VERB}{\Verb[commandchars=\\\{\}]}
\DefineVerbatimEnvironment{Highlighting}{Verbatim}{commandchars=\\\{\}}
% Add ',fontsize=\small' for more characters per line
\usepackage{framed}
\definecolor{shadecolor}{RGB}{248,248,248}
\newenvironment{Shaded}{\begin{snugshade}}{\end{snugshade}}
\newcommand{\AlertTok}[1]{\textcolor[rgb]{0.94,0.16,0.16}{#1}}
\newcommand{\AnnotationTok}[1]{\textcolor[rgb]{0.56,0.35,0.01}{\textbf{\textit{#1}}}}
\newcommand{\AttributeTok}[1]{\textcolor[rgb]{0.77,0.63,0.00}{#1}}
\newcommand{\BaseNTok}[1]{\textcolor[rgb]{0.00,0.00,0.81}{#1}}
\newcommand{\BuiltInTok}[1]{#1}
\newcommand{\CharTok}[1]{\textcolor[rgb]{0.31,0.60,0.02}{#1}}
\newcommand{\CommentTok}[1]{\textcolor[rgb]{0.56,0.35,0.01}{\textit{#1}}}
\newcommand{\CommentVarTok}[1]{\textcolor[rgb]{0.56,0.35,0.01}{\textbf{\textit{#1}}}}
\newcommand{\ConstantTok}[1]{\textcolor[rgb]{0.00,0.00,0.00}{#1}}
\newcommand{\ControlFlowTok}[1]{\textcolor[rgb]{0.13,0.29,0.53}{\textbf{#1}}}
\newcommand{\DataTypeTok}[1]{\textcolor[rgb]{0.13,0.29,0.53}{#1}}
\newcommand{\DecValTok}[1]{\textcolor[rgb]{0.00,0.00,0.81}{#1}}
\newcommand{\DocumentationTok}[1]{\textcolor[rgb]{0.56,0.35,0.01}{\textbf{\textit{#1}}}}
\newcommand{\ErrorTok}[1]{\textcolor[rgb]{0.64,0.00,0.00}{\textbf{#1}}}
\newcommand{\ExtensionTok}[1]{#1}
\newcommand{\FloatTok}[1]{\textcolor[rgb]{0.00,0.00,0.81}{#1}}
\newcommand{\FunctionTok}[1]{\textcolor[rgb]{0.00,0.00,0.00}{#1}}
\newcommand{\ImportTok}[1]{#1}
\newcommand{\InformationTok}[1]{\textcolor[rgb]{0.56,0.35,0.01}{\textbf{\textit{#1}}}}
\newcommand{\KeywordTok}[1]{\textcolor[rgb]{0.13,0.29,0.53}{\textbf{#1}}}
\newcommand{\NormalTok}[1]{#1}
\newcommand{\OperatorTok}[1]{\textcolor[rgb]{0.81,0.36,0.00}{\textbf{#1}}}
\newcommand{\OtherTok}[1]{\textcolor[rgb]{0.56,0.35,0.01}{#1}}
\newcommand{\PreprocessorTok}[1]{\textcolor[rgb]{0.56,0.35,0.01}{\textit{#1}}}
\newcommand{\RegionMarkerTok}[1]{#1}
\newcommand{\SpecialCharTok}[1]{\textcolor[rgb]{0.00,0.00,0.00}{#1}}
\newcommand{\SpecialStringTok}[1]{\textcolor[rgb]{0.31,0.60,0.02}{#1}}
\newcommand{\StringTok}[1]{\textcolor[rgb]{0.31,0.60,0.02}{#1}}
\newcommand{\VariableTok}[1]{\textcolor[rgb]{0.00,0.00,0.00}{#1}}
\newcommand{\VerbatimStringTok}[1]{\textcolor[rgb]{0.31,0.60,0.02}{#1}}
\newcommand{\WarningTok}[1]{\textcolor[rgb]{0.56,0.35,0.01}{\textbf{\textit{#1}}}}
\usepackage{graphicx}
\makeatletter
\def\maxwidth{\ifdim\Gin@nat@width>\linewidth\linewidth\else\Gin@nat@width\fi}
\def\maxheight{\ifdim\Gin@nat@height>\textheight\textheight\else\Gin@nat@height\fi}
\makeatother
% Scale images if necessary, so that they will not overflow the page
% margins by default, and it is still possible to overwrite the defaults
% using explicit options in \includegraphics[width, height, ...]{}
\setkeys{Gin}{width=\maxwidth,height=\maxheight,keepaspectratio}
% Set default figure placement to htbp
\makeatletter
\def\fps@figure{htbp}
\makeatother
\setlength{\emergencystretch}{3em} % prevent overfull lines
\providecommand{\tightlist}{%
  \setlength{\itemsep}{0pt}\setlength{\parskip}{0pt}}
\setcounter{secnumdepth}{-\maxdimen} % remove section numbering
\ifluatex
  \usepackage{selnolig}  % disable illegal ligatures
\fi

\title{Latihan 2}
\author{Hamzah Abdulloh}
\date{9/19/2021}

\begin{document}
\maketitle

A. \textbf{Tujuan Praktikum}

Memahami jenis-jenis tipe data pada R.

B. \textbf{Alokasi Waktu}

1 x pertemuam = 120 menit

C. \textbf{Dasar Teori}

Variasi tipe data pada R memfasilitasi keberagaman jenis variabel data.
Sebagai contoh, terdapat data yang terdiri dari sekumpulan angka dan
data lain yang berisi sekumpulan karakter. Pada contoh lain, ada pula
data yang berbentuk tabel maupun kumpulan (\emph{list}) angka sederhana.
Dengan bantuan fungsi \emph{class}, kita akan mendapatkan kemudahan
dalam mendefinisikan tipe data yang kita miliki:

\begin{Shaded}
\begin{Highlighting}[]
\NormalTok{a }\OtherTok{\textless{}{-}} \DecValTok{2}
\FunctionTok{class}\NormalTok{(a)}
\CommentTok{\#\textgreater{} [1] "numeric"}
\end{Highlighting}
\end{Shaded}

Agar dapat bekerja secara efisien dalam menggunakan bahasa pemrograman
R, penting untuk mempelajari terlebih dahulu tipe data dari
variabel-variabel yang kita miliki sehingga akan mempermudah dalam
penentuan proses analisis data yang dapat dilakukan terhadap variabel-
variabel tersebut

\textbf{\emph{Data Frame}}

Cara paling umum yang dapat digunakan untuk menyimpan \emph{dataset}
dalam R adalah dalam tipe data \emph{frame}. Secara konseptual, kita
dapat menganggap data frame sebagai tabel yang terdiri dari baris yang
memiliki nilai pengamatan dan berbagai variabel yang didefinisikan dalam
bentuk kolom. Tipe data ini sangat umum digunakan untuk \emph{dataset},
karena data \emph{frame} dapat menggabungkan berbagai jenis tipe data
dalam satu objek. Untuk memahami tipe data frame, silahkan mengakses
contoh dataset pada \emph{library(dslabs)} dan pilih \emph{dataset
``murders''} menggunakan fungsi \emph{data}:

\begin{Shaded}
\begin{Highlighting}[]
\FunctionTok{library}\NormalTok{(dslabs) }
\FunctionTok{data}\NormalTok{(murders) }
\end{Highlighting}
\end{Shaded}

Untuk memastikan bahwa \emph{dataset} tersebut tipenya adalah \emph{data
frame}, dapat digunakan perintah berikut:

\begin{Shaded}
\begin{Highlighting}[]
\FunctionTok{class}\NormalTok{(murders) }
\CommentTok{\#\textgreater{} [1] "data.frame"  }
\end{Highlighting}
\end{Shaded}

Untuk memeriksa lebih lanjut isi \emph{dataset}, dapat pula digunakan
fungsi \emph{str} untuk mencari tahu lebih rinci mengenai struktur suatu
objek:

\begin{Shaded}
\begin{Highlighting}[]
\FunctionTok{str}\NormalTok{(murders) }
\CommentTok{\#\textgreater{} \textquotesingle{}data.frame\textquotesingle{}: 51 obs. of 5 variables: }
\CommentTok{\#\textgreater{} $ state : chr "Alabama" "Alaska" "Arizona" "Arkansas" ... }
\CommentTok{\#\textgreater{} $ abb : chr "AL" "AK" "AZ" "AR" ... }
\CommentTok{\#\textgreater{} $ region : Factor w/ 4 levels "Northeast","South",..: 2 4 4 2 4 4 1 2 2 }
\CommentTok{\#\textgreater{} 2 ... }
\CommentTok{\#\textgreater{} $ population: num 4779736 710231 6392017 2915918 37253956 ... }
\CommentTok{\#\textgreater{} $ total : num 135 19 232 93 1257 ... }
\end{Highlighting}
\end{Shaded}

Dengan menggunakan fungsi \emph{str}, dapat diketahui bahwa
\emph{dataset ``murders''} terdiri dari 51 baris dan lima variabel:
\emph{state, abb, region, population,} dan \emph{total}. Selanjutnya,
untuk melihat contoh enam baris pertama pada \emph{dataset}, dapat
digunakan fungsi \emph{head}:

\begin{Shaded}
\begin{Highlighting}[]
\FunctionTok{head}\NormalTok{(murders) }
\CommentTok{\#\textgreater{} state abb region population total }
\CommentTok{\#\textgreater{} 1 Alabama AL South 4779736 135 }
\CommentTok{\#\textgreater{} 2 Alaska AK West 710231 19 }
\CommentTok{\#\textgreater{} 3 Arizona AZ West 6392017 232 }
\CommentTok{\#\textgreater{} 4 Arkansas AR South 2915918 93 }
\CommentTok{\#\textgreater{} 5 California CA West 37253956 1257 }
\CommentTok{\#\textgreater{} 6 Colorado CO West 5029196 65 }
\end{Highlighting}
\end{Shaded}

Untuk analisis awal tiap variabel yang diwakili dalam bentuk kolom pada
tipe \emph{data frame}, dapat digunakan operator aksesor (\$) dengan
cara berikut:

\begin{Shaded}
\begin{Highlighting}[]
\NormalTok{murders}\SpecialCharTok{$}\NormalTok{population }
\CommentTok{\#\textgreater{} [1] 4779736 710231 6392017 2915918 37253956 5029196 3574097 }
\CommentTok{\#\textgreater{} [8] 897934 601723 19687653 9920000 1360301 1567582 12830632 }
\CommentTok{\#\textgreater{} [15] 6483802 3046355 2853118 4339367 4533372 1328361 5773552 }
\CommentTok{\#\textgreater{} [22] 6547629 9883640 5303925 2967297 5988927 989415 1826341 }
\CommentTok{\#\textgreater{} [29] 2700551 1316470 8791894 2059179 19378102 9535483 672591 }
\CommentTok{\#\textgreater{} [36] 11536504 3751351 3831074 12702379 1052567 4625364 814180 }
\CommentTok{\#\textgreater{} [43] 6346105 25145561 2763885 625741 8001024 6724540 1852994 }
\CommentTok{\#\textgreater{} [50] 5686986 563626 }
\end{Highlighting}
\end{Shaded}

Untuk mengetahui nama-nama dari lima variabel yang dapat dievaluasi
menggunakan operator aksesor, sebelumnya, melalui fungsi \emph{str},
telah kita ketahui bahwa variabel yang dimiliki \emph{dataset} adalah:
\emph{state, abb, region, population,} dan \emph{total}. Sebagai
alternatif, terdapat pula fungsi \emph{name}, yang dapat digunakan
seperti contoh dibawah ini:

\begin{Shaded}
\begin{Highlighting}[]
\FunctionTok{names}\NormalTok{(murders) }
\CommentTok{\#\textgreater{} [1] "state" "abb" "region" "population" "total" }
\end{Highlighting}
\end{Shaded}

\textbf{\emph{Vector : numeric, character, dan logical}}

objek
\emph{murders\(population* terdiri dari sekumpulan numeric atau data-data angka. Sehingga, kita dapat mendefinisikan bahwa tipe data *murders\)population}
berupa \emph{vector}. Angka tunggal secara teknis dapat didefinisikan
sebagai vektor dengan panjang 1, tetapi secara umum kita akan\\
menggunakan \emph{vector} sebagai istilah untuk merujuk ke objek yang
terdiri dari beberapa entri. Untuk mengidentifikasi banyaknya entri
dalam suatu \emph{vector} dapat digunakan fungsi \emph{length} seperti
contoh berikut:

\begin{Shaded}
\begin{Highlighting}[]
\FunctionTok{length}\NormalTok{(murders}\SpecialCharTok{$}\NormalTok{population) }
\CommentTok{\#\textgreater{} [1] 51 }
\end{Highlighting}
\end{Shaded}

\emph{Vector} khusus ini bertipe \emph{numeric} karena populasi terdiri
dari data-data angka:

\begin{Shaded}
\begin{Highlighting}[]
\FunctionTok{class}\NormalTok{(murders}\SpecialCharTok{$}\NormalTok{population) }
\CommentTok{\#\textgreater{} [1] "numeric" }
\end{Highlighting}
\end{Shaded}

Secara matematis, nilai-nilai dalam \emph{murders\$population} adalah
berupa \emph{integer}. Namun, secara default, data angka akan diberikan
tipe \emph{numeric} meskipun sebenarnya data tersebut merupakan bilangan
bulat.Misalnya, \emph{class(1)} akan mengidentifikasi nilai 1 sebagai
tipe \emph{numeric}. Untuk mengubah tipe \emph{numeric} menjadi
\emph{integer}, dapat digunakan fungsi \emph{as.integer()} atau dengan
menambahkan L pada akhir data angka, contoh: 1L. Untuk melihat
perbedaannya, silahkan gunakan \emph{class(1L).}

Vector juga dapat digunakan untuk menyimpan \emph{string} dengan tipe
\emph{character}, Sebagai contoh: nama negara pada \emph{dataset
``murders''}:

\begin{Shaded}
\begin{Highlighting}[]
\FunctionTok{class}\NormalTok{(murders}\SpecialCharTok{$}\NormalTok{state) }
\CommentTok{\#\textgreater{} [1] "character"}
\end{Highlighting}
\end{Shaded}

Jenis vector penting lainnya adalah \emph{logical} yang nilainya berupa
TRUE atau FALSE.

\begin{Shaded}
\begin{Highlighting}[]
\NormalTok{z }\OtherTok{\textless{}{-}} \DecValTok{3} \SpecialCharTok{==} \DecValTok{2} 
\NormalTok{z }
\CommentTok{\#\textgreater{} [1] FALSE }
\FunctionTok{class}\NormalTok{(z) }
\CommentTok{\#\textgreater{} [1] "logical"}
\end{Highlighting}
\end{Shaded}

\textbf{\emph{Factor}}

Dalam \emph{dataset ``murders''}, variabel \emph{state} yang berisi data
karakter bukan bertipe \emph{vector: character,} namun, tipe datanya
adalah \emph{factor}:

\begin{Shaded}
\begin{Highlighting}[]
\FunctionTok{class}\NormalTok{(murders}\SpecialCharTok{$}\NormalTok{region) }
\CommentTok{\#\textgreater{} [1] "factor" }
\end{Highlighting}
\end{Shaded}

Faktor berguna untuk menyimpan data kategorikal. Dapat dilihat, bahwa
hanya terdapat 4 wilayah pada variabel \emph{state}. Untuk melihat
jumlah kategori yang dimiliki oleh variabel dengan tipe data
\emph{factor} dapat digunakan fungsi \emph{level}:

\begin{Shaded}
\begin{Highlighting}[]
\FunctionTok{levels}\NormalTok{(murders}\SpecialCharTok{$}\NormalTok{region) }
\CommentTok{\#\textgreater{} [1] "Northeast" "South" "North Central" "West"}
\end{Highlighting}
\end{Shaded}

Pada \emph{background process}, R menyimpan level sebagai bilangan bulat
yang memiliki peta tersendiri untuk melacak arti label dari bilangan
tersebut. Hal ini dimaksudkan untuk penghematan memori, terutama apabila
karakter dari tiap level cukup panjang. Standarnya, level akan
ditampilkan sesuai urutan abjad.

\textbf{\emph{Lists}} Data frame merupakan sekumpulan \emph{list} yang
memiliki kelas yang berbeda-beda. Sama halnya dengan data \emph{frame},
analisis \emph{list} dapat dilakukan dengan menggunakan operator aksesor
(\$) dan dua kurung siku ({[}{[}).

Matriks Matriks merupakan tipe data yang mirip dengan \emph{data frame}
karena keduanya memiliki dua dimensi, yaitu: baris dan kolom. Namun,
sama halnya dengan tipe data \emph{vector} numerik, karakter dan logis,
entri dalam matriks harus terdiri dari jenis \emph{vector} yang sama.
Dalam hal ini, \emph{data frame} dapat dikatakan sebagai tipe data yang
paling cocok untuk menyimpan data, karena kita dapat memiliki karakter,
faktor, dan angka sekaligus dalam satu data \emph{frame}. Namun matriks
memiliki satu keunggulan yang tidak dimiliki oleh tipe data frame: pada
matriks dapat dilakukan operasi aljabar. Untuk mendefinisikan matriks,
dapat digunakan fungsi matrix dengan mendefinisikan pula argumen berupa
jumlah baris dan kolom yang diinginkan.

\begin{Shaded}
\begin{Highlighting}[]
\NormalTok{mat }\OtherTok{\textless{}{-}} \FunctionTok{matrix}\NormalTok{(}\DecValTok{1}\SpecialCharTok{:}\DecValTok{12}\NormalTok{, }\DecValTok{4}\NormalTok{, }\DecValTok{3}\NormalTok{) }
\NormalTok{mat }
\CommentTok{\#\textgreater{}      [,1] [,2] [,3] }
\CommentTok{\#\textgreater{} [1,]   1    5    9 }
\CommentTok{\#\textgreater{} [2,]   2    6   10 }
\CommentTok{\#\textgreater{} [3,]   3    7   11 }
\CommentTok{\#\textgreater{} [4,]   4    8   12 }
\end{Highlighting}
\end{Shaded}

Untuk mengakses entri tertentu dalam matriks, dapat digunakan tanda
kurung siku ({[}). Sebagai contoh, kita akan menampilkan data pada baris
kedua, kolom ketiga, menggunakan:

\begin{Shaded}
\begin{Highlighting}[]
\NormalTok{mat[}\DecValTok{2}\NormalTok{, }\DecValTok{3}\NormalTok{] }
\CommentTok{\#\textgreater{} [1] 10 }
\end{Highlighting}
\end{Shaded}


\end{document}
