% Options for packages loaded elsewhere
\PassOptionsToPackage{unicode}{hyperref}
\PassOptionsToPackage{hyphens}{url}
%
\documentclass[
]{article}
\usepackage{amsmath,amssymb}
\usepackage{lmodern}
\usepackage{ifxetex,ifluatex}
\ifnum 0\ifxetex 1\fi\ifluatex 1\fi=0 % if pdftex
  \usepackage[T1]{fontenc}
  \usepackage[utf8]{inputenc}
  \usepackage{textcomp} % provide euro and other symbols
\else % if luatex or xetex
  \usepackage{unicode-math}
  \defaultfontfeatures{Scale=MatchLowercase}
  \defaultfontfeatures[\rmfamily]{Ligatures=TeX,Scale=1}
\fi
% Use upquote if available, for straight quotes in verbatim environments
\IfFileExists{upquote.sty}{\usepackage{upquote}}{}
\IfFileExists{microtype.sty}{% use microtype if available
  \usepackage[]{microtype}
  \UseMicrotypeSet[protrusion]{basicmath} % disable protrusion for tt fonts
}{}
\makeatletter
\@ifundefined{KOMAClassName}{% if non-KOMA class
  \IfFileExists{parskip.sty}{%
    \usepackage{parskip}
  }{% else
    \setlength{\parindent}{0pt}
    \setlength{\parskip}{6pt plus 2pt minus 1pt}}
}{% if KOMA class
  \KOMAoptions{parskip=half}}
\makeatother
\usepackage{xcolor}
\IfFileExists{xurl.sty}{\usepackage{xurl}}{} % add URL line breaks if available
\IfFileExists{bookmark.sty}{\usepackage{bookmark}}{\usepackage{hyperref}}
\hypersetup{
  pdftitle={Modul 3},
  pdfauthor={Hamzah Abdulloh},
  hidelinks,
  pdfcreator={LaTeX via pandoc}}
\urlstyle{same} % disable monospaced font for URLs
\usepackage[margin=1in]{geometry}
\usepackage{color}
\usepackage{fancyvrb}
\newcommand{\VerbBar}{|}
\newcommand{\VERB}{\Verb[commandchars=\\\{\}]}
\DefineVerbatimEnvironment{Highlighting}{Verbatim}{commandchars=\\\{\}}
% Add ',fontsize=\small' for more characters per line
\usepackage{framed}
\definecolor{shadecolor}{RGB}{248,248,248}
\newenvironment{Shaded}{\begin{snugshade}}{\end{snugshade}}
\newcommand{\AlertTok}[1]{\textcolor[rgb]{0.94,0.16,0.16}{#1}}
\newcommand{\AnnotationTok}[1]{\textcolor[rgb]{0.56,0.35,0.01}{\textbf{\textit{#1}}}}
\newcommand{\AttributeTok}[1]{\textcolor[rgb]{0.77,0.63,0.00}{#1}}
\newcommand{\BaseNTok}[1]{\textcolor[rgb]{0.00,0.00,0.81}{#1}}
\newcommand{\BuiltInTok}[1]{#1}
\newcommand{\CharTok}[1]{\textcolor[rgb]{0.31,0.60,0.02}{#1}}
\newcommand{\CommentTok}[1]{\textcolor[rgb]{0.56,0.35,0.01}{\textit{#1}}}
\newcommand{\CommentVarTok}[1]{\textcolor[rgb]{0.56,0.35,0.01}{\textbf{\textit{#1}}}}
\newcommand{\ConstantTok}[1]{\textcolor[rgb]{0.00,0.00,0.00}{#1}}
\newcommand{\ControlFlowTok}[1]{\textcolor[rgb]{0.13,0.29,0.53}{\textbf{#1}}}
\newcommand{\DataTypeTok}[1]{\textcolor[rgb]{0.13,0.29,0.53}{#1}}
\newcommand{\DecValTok}[1]{\textcolor[rgb]{0.00,0.00,0.81}{#1}}
\newcommand{\DocumentationTok}[1]{\textcolor[rgb]{0.56,0.35,0.01}{\textbf{\textit{#1}}}}
\newcommand{\ErrorTok}[1]{\textcolor[rgb]{0.64,0.00,0.00}{\textbf{#1}}}
\newcommand{\ExtensionTok}[1]{#1}
\newcommand{\FloatTok}[1]{\textcolor[rgb]{0.00,0.00,0.81}{#1}}
\newcommand{\FunctionTok}[1]{\textcolor[rgb]{0.00,0.00,0.00}{#1}}
\newcommand{\ImportTok}[1]{#1}
\newcommand{\InformationTok}[1]{\textcolor[rgb]{0.56,0.35,0.01}{\textbf{\textit{#1}}}}
\newcommand{\KeywordTok}[1]{\textcolor[rgb]{0.13,0.29,0.53}{\textbf{#1}}}
\newcommand{\NormalTok}[1]{#1}
\newcommand{\OperatorTok}[1]{\textcolor[rgb]{0.81,0.36,0.00}{\textbf{#1}}}
\newcommand{\OtherTok}[1]{\textcolor[rgb]{0.56,0.35,0.01}{#1}}
\newcommand{\PreprocessorTok}[1]{\textcolor[rgb]{0.56,0.35,0.01}{\textit{#1}}}
\newcommand{\RegionMarkerTok}[1]{#1}
\newcommand{\SpecialCharTok}[1]{\textcolor[rgb]{0.00,0.00,0.00}{#1}}
\newcommand{\SpecialStringTok}[1]{\textcolor[rgb]{0.31,0.60,0.02}{#1}}
\newcommand{\StringTok}[1]{\textcolor[rgb]{0.31,0.60,0.02}{#1}}
\newcommand{\VariableTok}[1]{\textcolor[rgb]{0.00,0.00,0.00}{#1}}
\newcommand{\VerbatimStringTok}[1]{\textcolor[rgb]{0.31,0.60,0.02}{#1}}
\newcommand{\WarningTok}[1]{\textcolor[rgb]{0.56,0.35,0.01}{\textbf{\textit{#1}}}}
\usepackage{graphicx}
\makeatletter
\def\maxwidth{\ifdim\Gin@nat@width>\linewidth\linewidth\else\Gin@nat@width\fi}
\def\maxheight{\ifdim\Gin@nat@height>\textheight\textheight\else\Gin@nat@height\fi}
\makeatother
% Scale images if necessary, so that they will not overflow the page
% margins by default, and it is still possible to overwrite the defaults
% using explicit options in \includegraphics[width, height, ...]{}
\setkeys{Gin}{width=\maxwidth,height=\maxheight,keepaspectratio}
% Set default figure placement to htbp
\makeatletter
\def\fps@figure{htbp}
\makeatother
\setlength{\emergencystretch}{3em} % prevent overfull lines
\providecommand{\tightlist}{%
  \setlength{\itemsep}{0pt}\setlength{\parskip}{0pt}}
\setcounter{secnumdepth}{-\maxdimen} % remove section numbering
\ifluatex
  \usepackage{selnolig}  % disable illegal ligatures
\fi

\title{Modul 3}
\author{Hamzah Abdulloh}
\date{9/29/2021}

\begin{document}
\maketitle

\hypertarget{latihan-modul-3}{%
\subsection{Latihan Modul 3}\label{latihan-modul-3}}

Import Library

\begin{Shaded}
\begin{Highlighting}[]
\FunctionTok{library}\NormalTok{(dslabs)}
\FunctionTok{data}\NormalTok{(murders) }
\end{Highlighting}
\end{Shaded}

\hypertarget{gunakan-fungsi-str-untuk-memeriksa-struktur-objek-murders.-manakah-dari-pernyataan-berikut-ini-yang-paling-menggambarkan-karakter-dari-tiap-variabel-pada-data-frame}{%
\subsubsection{1. Gunakan fungsi str untuk memeriksa struktur objek
``murders''. Manakah dari pernyataan berikut ini yang paling
menggambarkan karakter dari tiap variabel pada data
frame?}\label{gunakan-fungsi-str-untuk-memeriksa-struktur-objek-murders.-manakah-dari-pernyataan-berikut-ini-yang-paling-menggambarkan-karakter-dari-tiap-variabel-pada-data-frame}}

\begin{enumerate}
\def\labelenumi{\alph{enumi}.}
\tightlist
\item
  Terdiri dari 51 negara.
\item
  Data berisi tingkat pembunuhan pada 50 negara bagian dan DC.
\item
  Data berisi Nama negara bagian, singkatan dari nama negara bagian,
  wilayah negara bagian, dan populasi negara bagian serta jumlah total
  pembunuhan pada tahun 2010.
\item
  str tidak menunjukkan informasi yang relevan.
\end{enumerate}

\textbf{Jawaban}

\begin{Shaded}
\begin{Highlighting}[]
\FunctionTok{str}\NormalTok{(murders)}
\end{Highlighting}
\end{Shaded}

\begin{verbatim}
## 'data.frame':    51 obs. of  5 variables:
##  $ state     : chr  "Alabama" "Alaska" "Arizona" "Arkansas" ...
##  $ abb       : chr  "AL" "AK" "AZ" "AR" ...
##  $ region    : Factor w/ 4 levels "Northeast","South",..: 2 4 4 2 4 4 1 2 2 2 ...
##  $ population: num  4779736 710231 6392017 2915918 37253956 ...
##  $ total     : num  135 19 232 93 1257 ...
\end{verbatim}

\textbf{Maka jawaban yang benar adalah ``C'' }

\hypertarget{sebutkan-apa-saja-nama-kolom-yang-digunakan-pada-data-frame}{%
\paragraph{2. Sebutkan apa saja nama kolom yang digunakan pada data
frame}\label{sebutkan-apa-saja-nama-kolom-yang-digunakan-pada-data-frame}}

.

\textbf{Jawaban}

\begin{Shaded}
\begin{Highlighting}[]
\FunctionTok{names}\NormalTok{(murders)}
\end{Highlighting}
\end{Shaded}

\begin{verbatim}
## [1] "state"      "abb"        "region"     "population" "total"
\end{verbatim}

\textbf{Nama kolom yang digunakan pada data frame adalah : state, abb,
region, pupulation dan total}

\hypertarget{gunakan-operator-aksesor-untuk-mengekstrak-informasi-singkatan-negara-dan-menyimpannya-pada-objek-a.-sebutkan-jenis-class-dari-objek-tersebut}{%
\paragraph{3. Gunakan operator aksesor (\$) untuk mengekstrak informasi
singkatan negara dan menyimpannya pada objek ``a''. Sebutkan jenis class
dari objek
tersebut}\label{gunakan-operator-aksesor-untuk-mengekstrak-informasi-singkatan-negara-dan-menyimpannya-pada-objek-a.-sebutkan-jenis-class-dari-objek-tersebut}}

.

\textbf{Jawaban}

\begin{Shaded}
\begin{Highlighting}[]
\NormalTok{a }\OtherTok{\textless{}{-}}\NormalTok{ murders}\SpecialCharTok{$}\NormalTok{abb}
\NormalTok{a}
\end{Highlighting}
\end{Shaded}

\begin{verbatim}
##  [1] "AL" "AK" "AZ" "AR" "CA" "CO" "CT" "DE" "DC" "FL" "GA" "HI" "ID" "IL" "IN"
## [16] "IA" "KS" "KY" "LA" "ME" "MD" "MA" "MI" "MN" "MS" "MO" "MT" "NE" "NV" "NH"
## [31] "NJ" "NM" "NY" "NC" "ND" "OH" "OK" "OR" "PA" "RI" "SC" "SD" "TN" "TX" "UT"
## [46] "VT" "VA" "WA" "WV" "WI" "WY"
\end{verbatim}

\begin{Shaded}
\begin{Highlighting}[]
\FunctionTok{class}\NormalTok{(a)}
\end{Highlighting}
\end{Shaded}

\begin{verbatim}
## [1] "character"
\end{verbatim}

\textbf{objek a merupakan character}

\hypertarget{gunakan-tanda-kurung-siku-untuk-mengekstrak-singkatan-negara-dan-menyimpannya-pada-objek-b.-tentukan-apakah-variabel-a-dan-b-bernilai-sama}{%
\paragraph{4. Gunakan tanda kurung siku untuk mengekstrak singkatan
negara dan menyimpannya pada objek ``b''. Tentukan apakah variabel ``a''
dan ``b'' bernilai
sama}\label{gunakan-tanda-kurung-siku-untuk-mengekstrak-singkatan-negara-dan-menyimpannya-pada-objek-b.-tentukan-apakah-variabel-a-dan-b-bernilai-sama}}

?

\textbf{Jawaban}

\begin{Shaded}
\begin{Highlighting}[]
\NormalTok{a }\OtherTok{\textless{}{-}}\NormalTok{ murders}\SpecialCharTok{$}\NormalTok{abb}
\NormalTok{b }\OtherTok{\textless{}{-}}\NormalTok{ murders[}\StringTok{\textquotesingle{}abb\textquotesingle{}}\NormalTok{]}
\end{Highlighting}
\end{Shaded}

variabel a

\begin{Shaded}
\begin{Highlighting}[]
\NormalTok{a}
\end{Highlighting}
\end{Shaded}

\begin{verbatim}
##  [1] "AL" "AK" "AZ" "AR" "CA" "CO" "CT" "DE" "DC" "FL" "GA" "HI" "ID" "IL" "IN"
## [16] "IA" "KS" "KY" "LA" "ME" "MD" "MA" "MI" "MN" "MS" "MO" "MT" "NE" "NV" "NH"
## [31] "NJ" "NM" "NY" "NC" "ND" "OH" "OK" "OR" "PA" "RI" "SC" "SD" "TN" "TX" "UT"
## [46] "VT" "VA" "WA" "WV" "WI" "WY"
\end{verbatim}

variabel b

\begin{Shaded}
\begin{Highlighting}[]
\NormalTok{b}
\end{Highlighting}
\end{Shaded}

\begin{verbatim}
##    abb
## 1   AL
## 2   AK
## 3   AZ
## 4   AR
## 5   CA
## 6   CO
## 7   CT
## 8   DE
## 9   DC
## 10  FL
## 11  GA
## 12  HI
## 13  ID
## 14  IL
## 15  IN
## 16  IA
## 17  KS
## 18  KY
## 19  LA
## 20  ME
## 21  MD
## 22  MA
## 23  MI
## 24  MN
## 25  MS
## 26  MO
## 27  MT
## 28  NE
## 29  NV
## 30  NH
## 31  NJ
## 32  NM
## 33  NY
## 34  NC
## 35  ND
## 36  OH
## 37  OK
## 38  OR
## 39  PA
## 40  RI
## 41  SC
## 42  SD
## 43  TN
## 44  TX
## 45  UT
## 46  VT
## 47  VA
## 48  WA
## 49  WV
## 50  WI
## 51  WY
\end{verbatim}

kelas a

\begin{Shaded}
\begin{Highlighting}[]
\FunctionTok{class}\NormalTok{(a)}
\end{Highlighting}
\end{Shaded}

\begin{verbatim}
## [1] "character"
\end{verbatim}

kelas b

\begin{Shaded}
\begin{Highlighting}[]
\FunctionTok{class}\NormalTok{(b)}
\end{Highlighting}
\end{Shaded}

\begin{verbatim}
## [1] "data.frame"
\end{verbatim}

\textbf{variabel a dan b memiliki isi yang sama namun dengan tipe yang
berbeda. a merupakan vector charachter sementa b adalah data frame}
\#\#\#\# 5. Variabel region memiliki tipe data: factor. Dengan satu
baris kode, gunakan fungsi level dan length untuk menentukan jumlah
region yang dimiliki dataset .

\textbf{Jawaban}

\begin{Shaded}
\begin{Highlighting}[]
\FunctionTok{length}\NormalTok{(murders}\SpecialCharTok{$}\NormalTok{region)}
\end{Highlighting}
\end{Shaded}

\begin{verbatim}
## [1] 51
\end{verbatim}

\begin{Shaded}
\begin{Highlighting}[]
\FunctionTok{levels}\NormalTok{(murders}\SpecialCharTok{$}\NormalTok{region) }
\end{Highlighting}
\end{Shaded}

\begin{verbatim}
## [1] "Northeast"     "South"         "North Central" "West"
\end{verbatim}

\textbf{Jumlah keseluruhan region yang dimiliki oleh dataset adalah 51
dengan 4 jenis diantaranya : ``Northeast'' ``South'' ``North Central''
``West'' } \#\#\#\# 6. Fungsi table dapat digunakan untuk ekstraksi data
pada tipe vektor dan menampilkan frekuensi dari setiap elemen. Dengan
menerapkan fungsi tersebut, dapat diketahui jumlah state pada tiap
region. Gunakan fungsi table dalam satu baris kode untuk menampilkan
tabel baru yang berisi jumlah state pada tiap region .

\textbf{Jawaban}

\begin{Shaded}
\begin{Highlighting}[]
\FunctionTok{table}\NormalTok{(murders}\SpecialCharTok{$}\NormalTok{region)}
\end{Highlighting}
\end{Shaded}

\begin{verbatim}
## 
##     Northeast         South North Central          West 
##             9            17            12            13
\end{verbatim}

\end{document}
